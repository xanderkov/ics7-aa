\chapter{Технологический часть}

В данном разделе приведены требования к программному обеспечению, средства реализации и листинга кода.

\section{Требования к программе}

Программное обеспечение должно удовлетворять следующим требованиям:
\begin{itemize}
	\item на вход подается строка;
	\item на выходе --- результат поиска в словаре;
	\item программа не должна аварийно завершаться при отсутствии в словаре.
\end{itemize}

\section{Средства реализации} 
Для реализации ПО был выбран язык программирования $Python$ \cite{pythonlang}. 
В данном языке есть все требующиеся инструменты для данной лабораторной работы.
В качестве среды разработки была выбрана среда VS Code \cite{vscode}, запуск происходил через команду python main.py.


\section{Реализации алгоритма}
В листинге \ref{lst:bfs} представлена реализация алгоритма поиска в словаре полным перебором.
\clearpage

\begin{center}
	\captionsetup{justification=raggedright,singlelinecheck=off}
	
	\begin{lstlisting}[label=lst:bfs,caption=Реализация алгоритма поиска полным перебором]
		def full_comb_search(self, key):
		k = 0
		keys = list(self.data.keys())
		for elem in keys:
		if key == elem:
		return self.data[elem]
		return -1
	\end{lstlisting}
\end{center}

\section{Тестовые данные}

В данном разделе приведена таблица с тестами (таблица \ref{table:ref1}). Применена методология черного ящика. Тесты для всех алгоритмов пройдены \textit{успешно}.
\begin{center}
	\captionsetup{justification=raggedright,singlelinecheck=off}
	\begin{table}[ht]
		\centering
		\caption{Таблица тестов}
		\label{table:ref1}
		\begin{tabular}{ |c|c|c|}
			\hline
			Входные данные    & Пояснение   	  & Результат    \\ 
			\hline
			средний			  & Средний элемент   & Ответ верный \\ \hline
			легкий 			  & Первый элемент    & Ответ верный \\ \hline
			не употребительный 		  & Последний элемент & Ответ верный \\ \hline
			балтика & Несуществующий элемент & Ответ верный (-1) \\ \hline
			1477 & Несуществующий элемент & Ответ верный (-1) \\ \hline
		\end{tabular}
	\end{table}
\end{center}

\section*{Вывод}
Написано и протестировано программное обеспечение для решения поставленной задачи.