\chapter*{Введение}
\addcontentsline{toc}{chapter}{Введение}

В процессе развития компьютерных систем количество обрабатываемых данных увеличивалось, вледствие чего множество операций над наборами данных стали выполняться очень долго, поскольку чаще всего это был обычный перебор. 
Это вызвало необходимость создать новые алгоритмы, которые решают поставленную задачу на порядок быстрее стандартного решения прямого обхода. 
В том числе это касается и словарей, в которых одной из основных операций является операция поиска.

Цель лабораторной работы --- описание поиска по словарю при ограничении на значение признака, заданном при помощи лингвистической переменной. 

Задачи данной лабораторной:
\begin{itemize}
	\item описать объект по варианту и его признак;
	\item провести анкетирование респондентов;
	\item описать алгоритм поиска в словаре объектов;
	\item описать структуру данных словаря;
	\item реализовать описанный алгоритм поиска в словаре.
\end{itemize}