\chapter{Исследовательская часть}
\section{Технические характеристики}
Тестирование выполнялось на устройстве со следующими техническими характеристиками:
\begin{itemize}
	\item Операционная система Pop!\_OS 22.04 LTS \cite{ubuntu} Linux \cite{linux};
	\item Оперативная память 16 Гб;
	\item Процессор AMD® Ryzen 7 2700 eight-core processor × 16 \cite{amd}.
\end{itemize}

Во время тестирования устройство было подключено к блоку питания и не нагружено никакими приложениями, кроме встроенных приложений окружения, окружением и системой тестирования.

\section{Демонстрация работы программы}



На рисунке \ref{demonstration} представлен результат работы программы, в которой выводится исходный массив сгенерированный программой и три отсортированных массива.


\newpage

\section{Процессорное время выполнения реализации алгоритмов}

Результаты замеров времени работы реализаций алгоритмов сортировки на различных входных данных (в мс) приведены в таблицах \ref{tbl:best}, \ref{tbl:worth} и \ref{tbl:random}.
\newcolumntype{d}[1]{D{.}{.}{-1}}
\begin{table}[ht!]
	\begin{center}
			\captionsetup{justification=raggedright,singlelinecheck=off}
			\caption{Результаты замеров реализаций сортировок, входными данными явллялись отсортированные по возрастанию значений массивы.}
			\label{tbl:best}
			\begin{tabular}{|c|d{6.3}|d{6.3}|d{6.3}|}
				\hline
				Размер & \multicolumn{1}{c|}{\text{Подсчетом}} &  \multicolumn{1}{c|}{\text{Поразрядная}} & \multicolumn{1}{c|}{\text{Бинарным деревом}}  \\
				\hline
				100 & 0.1662 & 0.0714 & 0.8650 \\ 
				\hline
				200 & 0.5113 & 0.2058 & 3.3357 \\ 
				\hline
				300 & 1.1026 & 0.3131 & 7.6464 \\ 
				\hline
				400 & 2.0140 & 0.4364 & 13.6841 \\ 
				\hline
				500 & 3.3046 & 0.5591 & 21.5524 \\ 
				\hline
				600 & 5.0567 & 0.6798 & 31.3052 \\ 
				\hline
				700 & 6.6944 & 0.7852 & 43.0406 \\ 
				\hline
				800 & 8.5163 & 0.8766 & 56.4318 \\ 
				\hline
			\end{tabular}
	\end{center}
\end{table}

\newcolumntype{d}[1]{D{.}{.}{-1}}
\begin{table}[ht!]
	\begin{center}
			\captionsetup{justification=raggedright,singlelinecheck=off}
			\caption{Результаты замеров реализаций сортировок, входными данными явллялись отсортированные по убыванию значений массивы.}
			\label{tbl:worth}
			\begin{tabular}{|c|d{6.3}|d{6.3}|d{6.3}|}
	\hline
	Размер & \multicolumn{1}{c|}{\text{Подсчетом}} &  \multicolumn{1}{c|}{\text{Поразрядная}} & \multicolumn{1}{c|}{\text{Бинарным деревом}}  \\
	\hline
				100 & 0.1606 & 0.1048 & 0.7138 \\ 
				\hline
				200 & 0.5005 & 0.2008 & 2.7633 \\ 
				\hline
				300 & 1.0747 & 0.3110 & 6.3060 \\ 
				\hline
				400 & 1.9383 & 0.4312 & 11.3831 \\ 
				\hline
				500 & 3.1148 & 0.5427 & 18.0577 \\ 
				\hline
				600 & 4.6409 & 0.6693 & 26.0260 \\ 
				\hline
				700 & 6.7969 & 0.8317 & 36.7397 \\ 
				\hline
				800 & 8.7922 & 0.9583 & 47.2628 \\ 
				\hline
			\end{tabular}
	\end{center}
\end{table}

\newpage
\newcolumntype{d}[1]{D{.}{.}{-1}}
\begin{table}[ht!]
	\begin{center}
			\captionsetup{justification=raggedright,singlelinecheck=off}
			\caption{Результаты замеров реализаций сортировок, входными данными явллялись заполненные числами со случайными значениями массивы.}
			\label{tbl:random}
			\begin{tabular}{|c|d{6.3}|d{6.3}|d{6.3}|}
	\hline
	Размер & \multicolumn{1}{c|}{\text{Подсчетом}} &  \multicolumn{1}{c|}{\text{Поразрядная}} & \multicolumn{1}{c|}{\text{Бинарным деревом}}  \\
	\hline
				100 & 0.2734 & 0.1043 & 0.1560 \\ 
				\hline
				200 & 0.8321 & 0.2090 & 0.3756 \\ 
				\hline
				300 & 1.6837 & 0.3142 & 0.6025 \\ 
				\hline
				400 & 2.8938 & 0.4281 & 0.9785 \\ 
				\hline
				500 & 4.4438 & 0.5419 & 1.1784 \\ 
				\hline
				600 & 6.4153 & 0.6704 & 1.5523 \\ 
				\hline
				700 & 8.6692 & 0.7678 & 1.9018 \\ 
				\hline
				800 & 9.3752 & 0.8992 & 2.2986 \\ 
				\hline
			\end{tabular}
	\end{center}
\end{table}


\section{Результаты выполнения реализаций алгоритмов}

На графике \ref{graph:r} представлено время работы сортировок, входными данными явллялись заполненные числами со случайными значениями массивы.



На графике \ref{graph:r} представлено время работы сортировок, входными данными являлись отсортированные по возрастанию значений массивы.



На графике \ref{graph:r} представлено время работы сортировок, входными данными являлись отсортированные по убыванию значений массивы.



\newpage

\section*{Вывод}

В данном разделе были сравнены алгоритмы по времени.
Сортировка бинарным деревом на отсортированных массивах и сортировка подсчетом работает на случайном массиве дольше всех.

Самая быстрая сортировка поразрядная на любых данных.


Теоритические результаты оценки трудоемкости и полученные практическим образом результаты замеров совпадают. 
