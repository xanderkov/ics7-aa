\chapter{Технологический раздел}

В данном разделе будут приведены требования к программному обеспечению, средства реализации и листинга кода.

\section{Требования к программе}

Программное обеспечение должно удовлетворять следующим требованиям:
\begin{itemize}
	\item на вход подается имя файла;
	\item возвращается число кластеров;
	\item в программе возможно измерение реального времени.
\end{itemize}

\section{Средства реализации} 
Для реализации ПО был выбран язык программирования rust\cite{python}. 

В данном языке есть все требующиеся инструменты для данной лабораторной работы.

В качестве среды разработки была выбрана среда VS Code\cite{vscode}, запуск происходил через команду cargo run.

\section{Средства замера времени}

Замеры времени выполнения реализаций алгоритма будут проводиться при помощи функции Instant::now() \cite{test} библиотеки std::time. Данная команда возвращает значения  времени типа int в наносекундах.

\newpage

\begin{lstlisting}[label=bench,caption=Пример замера затраченного времени]
pub fn run_tests(points: Vec<Vec<bool>>, min_ptx: usize, eps: f64, imgbuf: RgbImage) {
	let n = 100;
	let img_guard = imgbuf.clone();
	for (algorithm, description) in MULTS_ARRAY.iter().zip(MULTS_DESCRIPTIONS.iter()) {
		let time = Instant::now();
		let mut result = 0;
		for _ in 0..n {
			let img_clone = img_guard.clone();
			result = algorithm(&points, min_ptx, eps, img_clone);
		}
		let time = time.elapsed().as_nanos();
	}
}
\end{lstlisting}



\section{Реализации алгоритма}
На листинге \ref{lst:algebr} демонстрируется реализация простого алгоритма DBSCAN. 
\newpage

\captionsetup{singlelinecheck = false, justification=raggedright}
\begin{lstlisting}[label=lst:algebr,caption=Реализация простого алгоритма DBSCAN]
pub fn dbscan(points: &Vec<Vec<bool>>, min_ptx: usize, eps: f64, mut imgbuf: RgbImage) -> u32 {
	let mut cluster_count = 0;
	let mut current_point = points.clone();
	let mut current_color = get_random_color();
	for i in 0..points.len() {
		for j in 0..points[i].len() {
			if current_point[i][j] {
				let mut v = Vec::<[usize; 2]>::new();
				v.push([i, j]);
				let mut neighbor_count_check = 0;
				while !v.is_empty() {
					let p = v.pop().unwrap();
					if !current_point[p[0]][p[1]] {
						continue;
					}
					current_point[p[0]][p[1]] = false;
					
					let mut neighbor_count = 0;
					
					regionquery(points, min_ptx, &mut v, &mut neighbor_count, p, eps);
					
					if neighbor_count >= min_ptx {
						neighbor_count_check += 1;
						*imgbuf.get_pixel_mut(p[1] as u32, p[0] as u32) = image::Rgb(current_color);
					}
				}
				if neighbor_count_check > 0 {
					cluster_count += 1;
					current_color = get_random_color();
				}
				current_point[i][j] = false;
			}
		}
	}
	
	cluster_count
}
\end{lstlisting}


На листинге \ref{lst:winograd} демонстрируется реализация функции поиска ближайших точек regionquery.

\begin{lstlisting}[label=lst:winograd,caption=Реализация функции regionquery]
fn regionquery(points: &Vec<Vec<bool>>, min_ptx: usize, v: &mut Vec::<[usize; 2]>, neighbor_count: &mut usize, p: [usize; 2], eps: f64)  {
	let start_x = cmp::max(0, p[0] - min_ptx);
	let start_y = cmp::max(0, p[1] - min_ptx);
	let end_x = cmp::min(points.len(), p[0] + min_ptx + 1);
	let end_y = cmp::min(points[0].len(), p[1] + min_ptx + 1);
	for _i in start_x..end_x {
		//check how many neighbors in the distance
		for _j in start_y..end_y {
			let distance = get_eculid_distance(_i as i32, _j as i32, p[0] as i32, p[1] as i32);
			if distance <= eps && points[_i][_j] {
				*neighbor_count += 1;
				v.push([_i, _j]);
			}
		}
	}
}
\end{lstlisting}	

На листинге \ref{lst:Tree} демонстрируется реализация параллельного алгоритма DBSCAN.

\newpage

\begin{lstlisting}[label=lst:Tree,caption=Реализация параллельного алгоритма DBSCAN]
pub fn dbscan_parallel(points: &Vec<Vec<bool>>, 
min_ptx: usize, eps: f64, 
range: std::ops::Range<usize>, 
guard_copy: Arc<Mutex<Vec<Vec<bool>>>>, 
n: Arc<Mutex<u32>>,
mut imgbuf: RgbImage) 
{
	let mut cluster_count = n.lock().unwrap();
	let mut current_point = guard_copy.lock().unwrap();
	let mut current_color = get_random_color();
	for i in range {
		for j in 0..points[i].len() {
			if current_point[i][j] {
				let mut v = Vec::<[usize; 2]>::new();
				v.push([i, j]);
				let mut neighbor_count_check = 0;
				while !v.is_empty() {
					let p = v.pop().unwrap();
					if !current_point[p[0]][p[1]] {
						continue;
					}
					current_point[p[0]][p[1]] = false;
					let mut neighbor_count = 0;
					regionquery(points, min_ptx, &mut v, &mut neighbor_count, p, eps);
					if neighbor_count >= min_ptx {
						neighbor_count_check += 1;
						*imgbuf.get_pixel_mut(p[1] as u32, p[0] as u32) = image::Rgb(current_color);
					}
				}
				if neighbor_count_check > 0 {
					*cluster_count += 1;
				}
				current_point[i][j] = false;
			}
		}
	}
	
}
\end{lstlisting}	

\newpage
На листинге \ref{lst:winograd-optimized} демонстрируется реализация функции параллелизации матрицы точек для алгоритма DBSCAN.
\begin{lstlisting}[label=lst:winograd-optimized,caption=Реализация функции параллелизации матрицы точек]
pub fn parrallize(points: &Vec<Vec<bool>>, min_ptx: usize, eps: f64, imgbuf: RgbImage, nofth: usize) -> u32 {
	
	let counter = Arc::new(Mutex::new(0));
	let current_point = Arc::new(Mutex::new(points.clone()));
	let size = points.len() / (nofth + 1);
	
	thread::scope(|s| {
		let mut threads = Vec::with_capacity(nofth);
		
		for i in 0..nofth {
			let range = (i * size)..((i + 1) * size);
			let guard_copy = current_point.clone();
			let counter_copy = counter.clone();
			let img_clone = imgbuf.clone();
			threads.push(s.spawn(move |_| dbscan_parallel(points, min_ptx, eps, range, guard_copy, counter_copy, img_clone)));
		}
		let range = (size * nofth)..points.len();
		let guard_\_timecopy = current_point.clone();
		let counter_copy = counter.clone();
		let img_clone = imgbuf.clone();
		dbscan_parallel(points, min_ptx, eps, range, guard_copy, counter_copy, img_clone);
		for th in threads {
			th.join().unwrap();
		}
		
		let counter_copy = counter.clone();
		let cluster_count = counter_copy.lock().unwrap();
		*cluster_count
	}).unwrap()
}
\end{lstlisting}	

\section{Тестовые данные}

В таблице \ref{tbl:functional_test} приведены тесты для функций, реализующих алгоритмы DBSCAN. Применена методология черного ящика. Тесты для всех алгоритмов пройдены \textit{успешно}.


\begin{table}[ht!]
	\begin{center}
		\captionsetup{justification=raggedright,singlelinecheck=off}
		\caption{\label{tbl:functional_test} Функциональные тесты}
		\begin{tabular}{|c|c|c|c|}
			\hline
			Вход & Ожидаемый результат & Результат послед. & Результат паралл. 
			\\ \hline
			$\begin{pmatrix}
				\\
			\end{pmatrix}$ &
			0&
			0 & 0
			
			\\ \hline
			$\begin{pmatrix}
			0\\
			\end{pmatrix}$ &
			0 &
			0 & 0
			
			\\ \hline
			$\begin{pmatrix}
				0 & 1
			\end{pmatrix}$ &
			1 &
			1 & 1
			
			\\ \hline
			$\begin{pmatrix}
				1
			\end{pmatrix}$ &
			1 &
			1 & 1
			
			\\ \hline
			$\begin{pmatrix}
				1 & 0 & 1\\
				0 & 0 & 0\\
				1 & 1 & 1
			\end{pmatrix}$ &
			3 &
			3 &3
			
			\\ \hline
			$\begin{pmatrix}
				1 & 0 \\
				0 & 1 
			\end{pmatrix}$ &
			2 &
			2 & 2               
			
			\\ \hline
			$\begin{pmatrix}
				1 & 0 & 0
			\end{pmatrix}$ &
			1&
			1 & 1
			
			\\ \hline
			$\begin{pmatrix}
				1 & 0 & 1 \\
				0 & 0 & 0 \\
				1 & 0 & 1
			\end{pmatrix}$ &
			4 &
			4  & 4                
			\\ \hline
		\end{tabular}
	\end{center}
\end{table}

\section*{Вывод}
Было написано и протестировано программное обеспечение для решения поставленной задачи.