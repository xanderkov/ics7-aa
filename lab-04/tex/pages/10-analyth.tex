\chapter{Аналитическая часть}

В этом разделе представляется описание алгоритма DBSCAN и его параллелизация.


\section{Плотностный алгоритм DBSCAN}

Алгоритм DBSCAN \cite{dbscan} (Density Based Spatial Clustering of Applications with Noise),
плотностный алгоритм для кластеризации пространственных данных с присутствием
шума). 
Данный алгоритм является решением разбиения (изначально пространственных) данных на кластеры произвольной формы \cite{cluster}.
Основная концепция алгоритма DBSCAN состоит в том, чтобы найти области высокой плотности, которые отделены друг от друга областями низкой насыщенности.
Кучность региона измеряется в два шага. 
Для каждой точки считается количество соседей в окружности радиуса $\epsilon$.
Далее определяется область плотности, где для каждой точки в кластере окружности с радиусом $\epsilon$ содержит больше минимального количество точек.

\section{Параллелизация плотностного алгоритма DBSCAN}

В изначальном плотностном алгоритме DBSCAB на вход подается множество точек, в виде матрицы, которая поэлементно обрабатывается в циклах.
Так как в алгоритме необходимо рассматривать точки и ее ближайших соседей, данный алгоритм возможно обрабатывать параллельно. 
Идея параллелизация состоит в том, чтобы обрабатывать не все множество, а некоторые его части разбитые по потокам.

\section*{Вывод}

Плотностный алгоритм DBSCAN независимо вычисляет элементы входного множества, что дает возможность реализовать параллельный вариант алгоритма.