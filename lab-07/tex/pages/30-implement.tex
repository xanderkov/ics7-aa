\chapter{Технологический часть}

В данном разделе приведены требования к программному обеспечению, средства реализации и листинга кода.

\section{Требования к программе}

Программное обеспечение должно удовлетворять следующим требованиям:
\begin{itemize}
	\item на вход подается имя файла;
	\item программа позволяет определять коэффициенты и количество дней;
	\item возможно измерение реального времени.
\end{itemize}

\section{Средства реализации} 
Для реализации ПО был выбран язык программирования Python\cite{python}. 

В данном языке есть все требующиеся инструменты для данной лабораторной работы.

В качестве среды разработки была выбрана среда VS Code\cite{vscode}, запуск происходил через команду python main.py.

\section{Средства замера времени}

Замеры времени выполнения реализаций алгоритма будут проводиться при помощи функции process\_time \cite{test} библиотеки time. Данная команда возвращает значения процессорного времени типа int в наносекундах.

Замеры времени для каждой реализации алгоритма и для каждого комплекта входных данных проводились 100 раз.
\newpage
\begin{lstlisting}[label=bench,caption=Пример замера затраченного времени]
	def test_simple_mult(A, B):
	# Start the stopwatch / counter 
	t1_start = process_time() 
	for i in range(N_TEST):
	simple_mult(A, M, B, N, M)
	# Stop the stopwatch / counter
	t1_stop = process_time()
\end{lstlisting}

\section{Реализации алгоритма}
В листинге \ref{lst:full_comb} представлен алгоритм полного перебора путей, а в листингах \ref{lst:ant_alg}--\ref{lst:upd_pher} --- муравьиный алгоритм и дополнительные к нему функции.


\begin{center}
	\captionsetup{justification=raggedright,singlelinecheck=off}
	\begin{lstlisting}[label=lst:full_comb,caption=Реализация алгоритма полного перебора путей]
	def fullCombinationAlg(matrix, size):
		places = np.arange(size)
		placesCombinations = list()
		
		for combination in it.permutations(places):
		combArr = list(combination)
		placesCombinations.append(combArr)
		minDist = float("inf")
		for i in range(len(placesCombinations)):
			placesCombinations[i].append(\
			placesCombinations[i][0])
			curDist = 0
			for j in range(size):
				startCity = placesCombinations[i][j]
				endCity = placesCombinations[i][j + 1]
				curDist += matrix[startCity][endCity]
			if (curDist < minDist):
				minDist = curDist
		bestWay = placesCombinations[i]
		return minDist, bestWay
	\end{lstlisting}
\end{center}


\begin{center}
	\captionsetup{justification=raggedright,singlelinecheck=off}
	\begin{lstlisting}[label=lst:ant_alg,caption=Реализация муравьиного алгоритма]
	def antAlgorithm(matrix, places, alpha, beta, k_evaporation, days):
		q = calcQ(matrix, places)
		bestWay = []
		minDist = float("inf")
		pheromones = calcPheromones(places)
		visibility = calcVisibility(matrix, places)
		ants = places
		for day in range(days):
		route = np.arange(places)
		visited = calcVisitedPlaces(route, ants)
		for ant in range(ants):
		while (len(visited[ant]) != ants):
		pk = findWays(pheromones, visibility, visited, places, ant, alpha, beta)
		chosenPlace = chooseNextPlaceByPosibility(pk)
		visited[ant].append(chosenPlace - 1)
		
		visited[ant].append(visited[ant][0])
		
		curLength = calcLength(matrix, visited[ant])
		
		if (curLength < minDist):
		minDist = curLength
		bestWay = visited[ant]
		
		pheromones = updatePheromones(matrix, places, visited, pheromones, q, k_evaporation)
		
		return minDist, bestWay
	\end{lstlisting}
\end{center}


\clearpage


\begin{center}
	\captionsetup{justification=raggedright,singlelinecheck=off}
	\begin{lstlisting}[label=lst:prob_arr,caption=Реализация алгоритма нахождения массива вероятностей переходов в непосещенные города]
	def findWays(pheromones, visibility, visited, places, ant, alpha, beta):
		pk = [0] * places
		
		for place in range(places):
		if place not in visited[ant]:
		ant_place = visited[ant][-1]
		pk[place] = pow(pheromones[ant_place][place], alpha) * \
		pow(visibility[ant_place][place], beta)
		else:
		pk[place] = 0
		
		sum_pk = sum(pk)
		
		for place in range(places):
		pk[place] /= sum_pk
		
		return pk
	\end{lstlisting}
\end{center}

\begin{center}
	\captionsetup{justification=raggedright,singlelinecheck=off}
	\begin{lstlisting}[label=lst:pheromone_above_Zero,caption=Реализация алгоритма нахождения массива вероятностей переходов в непосещенные города]
		def calcPheromones(size):
		min_phero = 1
		pheromones = [[min_phero for i in range(size)] for j in range(size)]
		return pheromones
	\end{lstlisting}
\end{center}


\clearpage

\begin{center}
	\captionsetup{justification=raggedright,singlelinecheck=off}
	\begin{lstlisting}[label=lst:choose_next,caption=Реализация алгоритма выбора следующего города]
	def chooseNextPlaceByPosibility(pk):
		posibility = random()
		choice = 0
		chosenPlace = 0
		
		while ((choice < posibility) and (chosenPlace < len(pk))):
		choice += pk[chosenPlace]
		chosenPlace += 1
		
		return chosenPlace
	\end{lstlisting}
\end{center}


\begin{center}
	\captionsetup{justification=raggedright,singlelinecheck=off}
	\begin{lstlisting}[label=lst:upd_pher,caption=Реализация алгоритма обновления матрицы феромонов]
	def updatePheromones(matrix, places, visited, pheromones, q, k_evaporation):
		ants = places
		
		for i in range(places):
		for j in range(places):
		delta = 0
		for ant in range(ants):
		length = calcLength(matrix, visited[ant])
		delta += q / length
		
		pheromones[i][j] *= (1 - k_evaporation)
		pheromones[i][j] += delta
		if (pheromones[i][j] < MIN_PHEROMONE):
		pheromones[i][j] = MIN_PHEROMONE
		
		return pheromones
	\end{lstlisting}
\end{center}

\section{Функциональные тесты}

В таблице \ref{tbl:functional_test} приведены тесты для функций, реализующих алгоритмы сортировки. Применена методология черного ящика. Тесты для всех сортировок пройдены \textit{успешно}.

\begin{center}
	\captionsetup{justification=raggedright,singlelinecheck=off}
	\begin{longtable}[c]{|c|c|c|c|c|}
		\caption{Функциональные тесты\label{tbl:functional_test}} \\ \hline
		Матрица смежности & Ожидаемый результат & Результат программы \\
		\hline
		$ \begin{pmatrix}
			0 &  4 &  2 &  1 & 7 \\
			4 &  0 &  3 &  7 & 2 \\
			2 &  3 &  0 & 10 & 3 \\
			1 &  7 & 10 &  0 & 9 \\
			7 &  2 &  3 &  9 & 0
		\end{pmatrix}$ &
		15, [0, 2, 4, 1, 3, 0] &
		15, [0, 2, 4, 1, 3, 0] \\
		
		$ \begin{pmatrix}
			0 & 1 & 2 \\
			1 & 0 & 1 \\
			2 & 1 & 0	
		\end{pmatrix}$ &
		4, [0, 1, 2, 0] &
		4, [0, 1, 2, 0] \\
		
		$ \begin{pmatrix}
			0 & 15 & 19 & 20 \\
			15 &  0 & 12 & 13 \\
			19 & 12 &  0 & 17 \\
			20 & 13 & 17 &  0
		\end{pmatrix}$ &
		64, [0, 1, 2, 3, 0] &
		64, [0, 1, 2, 3, 0] \\
		\hline
	\end{longtable}
\end{center}

\section*{Вывод}
Написано и протестировано программное обеспечение для решения поставленной задачи.