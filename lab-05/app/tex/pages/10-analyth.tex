\chapter{Аналитическая часть}

В этом разделе представляется описание алгоритма DBSCAN и его параллелизация.


\section{Плотностный алгоритм DBSCAN}

Алгоритм DBSCAN \cite{dbscan} (Density Based Spatial Clustering of Applications with Noise),
плотностный алгоритм для кластеризации пространственных данных с присутствием
шума). 
Данный алгоритм является решением разбиения (изначально пространственных) данных на кластеры произвольной формы \cite{cluster}.
Большинство алгоритмов, производящих плоское разбиение, создают кластеры по форме близкие к окружности, так как минимизируют расстояние документов до центра кластера. %% документы исправить

Идея, положенная в основу алгоритма, заключается в том, что внутри каждого кластера наблюдается типичная плотность точек  (объектов), которая заметно выше, чем плотность снаружи кластера, а также плотность в областях с шумом ниже
плотности любого из кластеров. 
Ещё точнее, что для каждой точки кластера её соседство заданного радиуса должно содержать не менее некоторого числа точек, это число точек задаётся пороговым значением.
%% помеять соседство только понято
\section{Параллелизация плотностного алгоритма DBSCAN}

В изначальном плотностном алгоритме DBSCAB на вход подается множество точек, в виде матрицы, которая поэлементно обрабатывается в циклах.
Так как в алгоритме необходимо рассматривать точки и ее ближайших соседей, данный алгоритм возможно обрабатывать параллельно. 
Идея параллелизация состоит в том, чтобы обрабатывать не все множество, а некоторые его части разбитые по потокам.

\section*{Вывод}

Плотностный алгоритм DBSCAN независимо вычисляет элементы входного множества, что дает возможность реализовать параллельный вариант алгоритма.