\chapter{Исследовательская часть}
\section{Технические характеристики}
Тестирование выполнялось на устройстве со следующими техническими характеристиками:
\begin{itemize}
	\item Операционная система Pop!\_OS 22.04 LTS \cite{ubuntu} Linux \cite{linux};
	\item Оперативная память 16 Гбайт;
	\item Процессор AMD® Ryzen 7 2700 eight-core processor × 16 \cite{amd}.
\end{itemize}

Во время тестирования устройство было подключено к блоку питания и не нагружено никакими приложениями, кроме встроенных приложений и системой тестирования.

\section{Демонстрация работы программы}



 Результат работы программы, в которой выводится время работы алгоритма и количество найденных кластеров представлены на рисунке \ref{demonstration}.



\newpage

\section{Время выполнения реализации алгоритмов}

Результаты замеров времени работы реализаций алгоритмов DBSCAN (в секундах) приведены в таблице \ref{tbl:best}. 
Сравнение проводилось между простым алгоритмом и параллельного алгоритма при исполнении на 8 потоках.
\newcolumntype{d}[1]{D{.}{.}{-1}}
\begin{table}[ht!]
	\begin{center}
			\captionsetup{justification=raggedright,singlelinecheck=off}
			\caption{Результаты замеров реализаций алгоритмов DBSCAN}
			\label{tbl:best}
			\begin{tabular}{|c|d{6.3}|d{6.3}|}
				\hline
				Размер & \multicolumn{1}{c|}{\text{Простой DBSCAN}} &  \multicolumn{1}{c|}{\text{Параллельный DBSCAN}}  \\
				\hline
				100 & 0.155203637 & 0.082057018 \\ 
				\hline
				200 & 1.886091723 & 1.183958677 \\ 
				\hline
				300 & 4.484127943 & 2.792452981 \\ 
				\hline
				400 & 5.975877116 & 3.666312837 \\ 
				\hline
				500 & 18.801009701 & 13.591029096
				\\
				\hline
			\end{tabular}
	\end{center}
\end{table}

На графике \ref{graph:r} представлены времена работы параллельной и последовательный реализаций алгоритмов DBSCAN.



На графике \ref{graph:j} представлена зависимоть времени работы параллельного алгоритма DBSCAN от количество потоков на квадратном изображении размера 500 пикселей.

\newpage

\section*{Вывод}

В данном разделе были сравнены алгоритмы по времени.
Параллельный алгоритм DBSCAN быстрее простого на 50 процентов (2.5 секунды) при размере 400 на 400.
Наилучшее время параллельный алгоритм показывает на 32 потоках.  
 
