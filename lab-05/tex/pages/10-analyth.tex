\chapter{Аналитическая часть}

В этом разделе представляется описание конвейерной обработки данных.


\section{Конвейерные вычисления}

Способ организации процесса в качестве вычислительного конвейера (pipeline) позволяет построить процесс, содержащий несколько независимых этапов \cite{pipeline}, на нескольких потоках. Выигрыш во времени достигается при выполнении нескольких задач за счет параллельной работы ступеней, вовлекая на каждом такте новую задачу или команду. 
Для контроля стадии используются три основные метрики, описанные ниже.
\begin{enumerate}
	\setlength{\itemsep}{1.2pt}
	\item Время процесса - это время, необходимое для выполнения одной стадии.
	\item Время выполнения - это время, которое требуется с момента, когда работа была выполнена на предыдущем этапе, до выполнения на текущем. 
	\item Время простоя - это время, когда никакой работы не происходит и линии простаивают. 
\end{enumerate}
Для того, чтобы время простоя было минимальным, стадии обработки должны быть одинаковы по времени в пределах погрешности. При возникновении ситуации, в которой время процесса одной из линий больше, чем время других в $N$ раз, эту линию стоит распараллелить на $N$ потоков. 



\section{Алгоритм DBSCAN}

Алгоритм DBSCAN \cite{dbscan} (Density Based Spatial Clustering of Applications with Noise),
плотностный алгоритм для кластеризации пространственных данных с присутствием
шума). 
Данный алгоритм является решением разбиения (изначально пространственных) данных на кластеры произвольной формы \cite{cluster}.
Основная концепция алгоритма DBSCAN состоит в том, чтобы найти области высокой плотности, которые отделены друг от друга областями низкой насыщенности.
Кучность региона измеряется в два шага. 
Для каждой точки считается количество соседей в окружности радиуса $\epsilon$.
Далее определяется область плотности, где для каждой точки в кластере окружности с радиусом $\epsilon$ содержит больше минимального количество точек.

\section*{Вывод}

Плотностный алгоритм DBSCAN работает независимо от чтения и вывода из файла, что дает возможность реализовать конвейер.