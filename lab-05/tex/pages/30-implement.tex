\chapter{Технологический часть}

В данном разделе приведены требования к программному обеспечению, средства реализации и листинга кода.

\section{Требования к программе}

Программное обеспечение должно удовлетворять следующим требованиям:
\begin{itemize}
	\item на вход подается имя файла;
	\item используется параллелизация программы;
	\item возможно измерение реального времени.
\end{itemize}

\section{Средства реализации} 
Для реализации ПО был выбран язык программирования Golang \cite{golang}. 
В данном языке есть все требующиеся инструменты для данной лабораторной работы.
В качестве среды разработки была выбрана среда VS Code \cite{vscode}, запуск происходил через команду go run main.go.


\section{Реализации алгоритма}
На листинге \ref{lst:algebr} приведена реализация простого алгоритма DBSCAN. 
\newpage

\captionsetup{singlelinecheck = false, justification=raggedright}
\begin{lstlisting}[label=lst:algebr,caption=Реализация простого алгоритма DBSCAN]
func DbscanAlgorithm(points structures.Matrix, minPtx int, eps float64) structures.Matrix {
	clusterCount := 0
	currentPoint := points
	img := structures.Allocate(points.Rows, points.Cols)
	color := 0
	for i := 0; i < points.Rows; i++ {
		for j:=0; j < points.Cols; j++ {
			if currentPoint.Values[i][j] == 1 {
				var v1 []int
				var v2 []int
				v1 = append(v1, i)
				v2 = append(v2, j)
				neighborCountCheck := 0
				for k := 0; k < len(v1); k++ {
					if currentPoint.Values[v1[k]][v2[k]] == 1 {
						currentPoint.Values[v1[k]][v2[k]] = 0
						neighborCount := 0
						_, startX := MinMax([]int{0, v1[k] - minPtx})
						_, startY := MinMax([]int{0, v2[k] - minPtx})
						endX, _ := MinMax([]int{points.Rows, v1[k] + minPtx + 1})
						endY, _ := MinMax([]int{points.Cols, v2[k] + minPtx + 1})
						if neighborCount >= minPtx {
							neighborCountCheck++
							img.Values[v1[k]][v2[k]] = color
						}
					}
				}
				if neighborCountCheck > 0 {
					clusterCount++
					color++
				}
				currentPoint.Values[i][j] = 0
			}
		}
	}
	return img
}
\end{lstlisting}


На листинге \ref{lst:winograd} приведена реализация функции поиска ближайших точек regionquery.

\begin{lstlisting}[label=lst:winograd,caption=Реализация функции regionquery]
for x := startX; x < endX; x++ {
	for y := startY; y < endY; y++ {
		distance := math.Sqrt(math.Pow(float64(startX - endX), 2) + math.Pow(float64(startY - endY), 2))
		
		if distance <= eps && currentPoint.Values[x][y] == 1 {
			neighborCount++
			
			v1 = append(v1, x)
			v2 = append(v2, y)
		}
	}
}
\end{lstlisting}	

На листинге \ref{lst:Tree} приведена реализация параллельного конвейера.


\begin{lstlisting}[label=lst:Tree,caption=Реализация параллельного алгоритма DBSCAN]
func Pipeline(count int, ch chan int, filename string, minPtx int, eps float64) *structures.Queue {
	first := make(chan *structures.PipeTask, count)
	second := make(chan *structures.PipeTask, count)
	third := make(chan *structures.PipeTask, count)
	line := InitQueue(count)
	get_file := func() {
		for {
			select {
				case pipe_task := <- first:
				pipe_task.Generated = true
				
				pipe_task.Start_generating = time.Now()
				
				pipe_task.Source = dbscan.ReadFile(filename)
				
				pipe_task.End_generatig = time.Now()
				
				second <- pipe_task
			}
		}
	}
	get_dbscan := func() {
		for {
			select {
				case pipe_task := <- second:
				pipe_task.Dbscan_mode = true
				
				pipe_task.Start_dbscan = time.Now()
				
				pipe_task.Dbscan = dbscan.DbscanAlgorithm(pipe_task.Source, minPtx, eps)
				
				pipe_task.End_dbscan = time.Now()
				
				third <- pipe_task
			}
		}
	}
	match := func ()  {
		for {
			select {
				case pipe_task := <- third:
				pipe_task.Pattern_matched = true
				
				pipe_task.Start_match = time.Now()
				pipe_task.Pattern_index = dbscan.SaveFile(pipe_task.Dbscan)
				pipe_task.End_match = time.Now()
				
				line.Enqueue(pipe_task)
				if (pipe_task.Num == count - 1) {
					ch <- 0
				}
			}
		}
	}
	go get_file()
	go get_dbscan()
	go match()
	
	for i := 0; i < count; i++ {
		pipe_task := new(structures.PipeTask)
		pipe_task.Num = i + 1
		first <- pipe_task
	}
	return line
}
\end{lstlisting}	

\newpage
На листинге \ref{lst:winograd-optimized} приведена реализация синхронного конвейера.
\begin{lstlisting}[label=lst:winograd-optimized,caption=Реализация функции параллелизации матрицы точек]
func gen_string_sync(task *structures.PipeTask) *structures.PipeTask {
	task.Generated = true
	task.Start_generating = time.Now()
	task.Source = dbscan.ReadFile("100.txt")
	task.End_generatig = time.Now()
	return task
}
func get_table_sync(task *structures.PipeTask) *structures.PipeTask {
	task.Dbscan_mode = true
	task.Start_dbscan = time.Now()
	task.Dbscan = dbscan.DbscanAlgorithm(task.Source, 2, 2)
	task.End_dbscan = time.Now()
	return task
}
func match_sync(task *structures.PipeTask) *structures.PipeTask {
	task.Pattern_matched = true
	task.Start_match = time.Now()
	task.Pattern_index = dbscan.SaveFile(task.Dbscan)
	task.End_match = time.Now()
	return task
}
func Sync(count int) *structures.Queue {
	line_first := InitQueue(count)
	line_second := InitQueue(count)
	line_third := InitQueue(count)
	for i := 0; i < count; i++ {
		pipe_task := new(structures.PipeTask)
		pipe_task = gen_string_sync(pipe_task)
		line_first.Enqueue(pipe_task)
		if (len(line_first.Queue) != 0) {
			pipe_task = get_table_sync(line_first.Dequeue())
			line_second.Enqueue(pipe_task)
			if (len(line_second.Queue) != 0) 
				pipe_task = match_sync(line_second.Dequeue())
				line_third.Enqueue(pipe_task)
			}
		}
	}
	return line_third
}
\end{lstlisting}	

\section{Тестовые данные}

В таблице \ref{tbl:functional_test} приведены тесты для функций, реализующих алгоритмы DBSCAN. Применена методология черного ящика. Тесты для всех алгоритмов пройдены \textit{успешно}.


\begin{table}[ht!]
	\begin{center}
		\captionsetup{justification=raggedright,singlelinecheck=off}
		\caption{\label{tbl:functional_test} Функциональные тесты}
		\begin{tabular}{|c|c|c|c|}
			\hline
			Вход & Ожидаемый результат & Результат послед. & Результат паралл. 
			\\ \hline
			$\begin{pmatrix}
				\\
			\end{pmatrix}$ &
			0&
			0 & 0
			
			\\ \hline
			$\begin{pmatrix}
			0\\
			\end{pmatrix}$ &
			0 &
			0 & 0
			
			\\ \hline
			$\begin{pmatrix}
				0 & 1
			\end{pmatrix}$ &
			1 &
			1 & 1
			
			\\ \hline
			$\begin{pmatrix}
				1
			\end{pmatrix}$ &
			1 &
			1 & 1
			
			\\ \hline
			$\begin{pmatrix}
				1 & 0 & 1\\
				0 & 0 & 0\\
				1 & 1 & 1
			\end{pmatrix}$ &
			3 &
			3 &3
			
			\\ \hline
			$\begin{pmatrix}
				1 & 0 \\
				0 & 1 
			\end{pmatrix}$ &
			2 &
			2 & 2               
			
			\\ \hline
			$\begin{pmatrix}
				1 & 0 & 0
			\end{pmatrix}$ &
			1&
			1 & 1
			
			\\ \hline
			$\begin{pmatrix}
				1 & 0 & 1 \\
				0 & 0 & 0 \\
				1 & 0 & 1
			\end{pmatrix}$ &
			4 &
			4  & 4                
			\\ \hline
		\end{tabular}
	\end{center}
\end{table}

\section*{Вывод}
Написано и протестировано программное обеспечение для решения поставленной задачи.