\chapter*{Введение}
\addcontentsline{toc}{chapter}{Введение}

Многопоточность \cite{mnogo} --- способность центрального процессора (CPU) или одного ядра в многоядерном процессоре одновременно выполнять несколько процессов или потоков, соответствующим образом поддерживаемых операционной системой.
Этот подход отличается от многопроцессорности, так как многопоточность процессов и потоков совместно использует ресурсы одного или нескольких ядер: вычислительных блоков, кэш-памяти ЦПУ или буфера перевода с преобразованием (TLB).

В тех случаях, когда многопроцессорные системы включают в себя несколько полных блоков обработки, многопоточность направлена на максимизацию использования ресурсов одного ядра, используя параллелизм на уровне потоков, а также на уровне инструкций.
Поскольку эти два метода являются взаимодополняющими, их иногда объединяют в системах с несколькими многопоточными ЦП и в ЦП с несколькими многопоточными ядрами.

Цель лабораторной работы --- исследование плотностного алгоритма DBSCAN.

\newpage

Задачи данной лабораторной:

\begin{itemize}
	\item описать понятие параллельных вычислений и плотностный алгоритм DBSCAN;
	\item реализовать последовательный и параллельный алгоритм DBSCAN;
	\item провести сравнительный анализ алгоритмов на основе экспериментальных данных, а именно по времени;
	\item описать и обосновать полученные результаты в отчете о выполненной лабораторной работе, выполненного как расчётно-пояснительная записка к работе.
\end{itemize}