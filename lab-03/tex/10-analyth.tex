\chapter{Аналитическая часть}

В этом разделе будут представлены описания алгоритмов сортировки бинарным деревом, подсчетом и поразрядная.


\section{Сортировка бинарным деревом}

\textbf{Сортировка бинарным деревом} \cite{binary} --- деревом назовем упорядоченную структуру данных, в которой каждому элементу --- предшественнику или корню (под)дерева --- поставлены в соответствие по крайней мере два других элемента (преемника). 
Причем для каждого предшественника выполнено следующее правило: левый преемник всегда меньше, а правый преемник всегда больше или равен предшественнику.  
Вместо 'предшественник' и 'преемник' также употребляют термины 'родитель' и 'сын'. Все элементы дерева также называют 'узлами'.

При добавлении в дерево нового элемента его последовательно сравнивают с нижестоящими узлами, таким образом вставляя на место.
Если элемент >= корня --- он идет в правое поддерево, сравниваем его уже с правым сыном, иначе --- он идет в левое поддерево, сравниваем с левым, и так далее, пока есть сыновья, с которыми можно сравнить.


\section{Сортировка подсчетом}

\textbf{Сортировка подсчетом \cite{counting_sort}} --- Сортировка подсчетом — это алгоритм сортировки на основе целых чисел для сортировки массива, ключи которого лежат в определенном диапазоне. Он подсчитывает общее количество элементов с каждым уникальным значением ключа, а затем использует эти подсчеты для определения позиций каждого значения ключа в выходных данных.

\section{Сортировка поразрядная}

\textbf{Сортировка поразрядная \cite{radix_sort}}. Массив несколько раз перебирается и элементы перегруппировываются в зависимости от того, какая цифра находится в определённом разряде. После обработки разрядов (всех или почти всех) массив оказывается упорядоченным.При этом разряды могут обрабатываться в противоположных направлениях - от младших к старшим или наоборот.



\section*{Вывод}

В данной работе стоит задача реализации 3 алгоритмов сортировки, а
именно: бинарным деревом, подсчетом и поразрядная. Необходимо оценить теоретическую оценку алгоритмов и проверить ее экспериментально.
