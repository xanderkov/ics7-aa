\chapter*{Введение}
\addcontentsline{toc}{chapter}{Введение}

Цель лабораторной работы --- разработка, реализация и исследование алгоритмов нахождения расстояний Левенштейна и Дамерау-Левенштейна.

Нахождение редакционного расстояния ~ одна из задач компьютерной лингвистики, которая находит применение в огромном количестве областей, начиная от предиктивных систем набора текста и заканчивая разработкой искусственного интеллекта. Впервые задачу поставил советский ученый В. И. Левенштейн \cite{Lev1965}, впоследствии её связали с его именем. В данной работе будут рассмотрены алгоритмы редакционного расстояния Левенштейна и расстояние Дамерау \,--\, Левенштейна \cite{damerau}.

Расстояния Левенштейна --- метрика, измеряющая разность двух строк символов, определяемая в количестве редакторских операций(а именно удаления, вставки и замены), требуемых для преобразования одной последовательности в другую.  Расстояние Дамерау -- Левенштейна ~ модификация, добавляющая к редакторским операциям транспозицию, или обмен двух соседних символов местами.
Алгоритмы имеют некоторое количество модификаций, позволяющих эффективнее решать поставленную задачу. В данной работе будут предложены реализации алгоритмов, использующие парадигмы динамического программирования. 


\newpage

Задачи данной лабораторной следующее:

\begin{enumerate}[label=\arabic*)]
	\item изучение расстояний Левенштейна и Дамерау-Левенштейна;
	
	\item применение метода динамического программирования для реализации алгоритма;
	
	\item получение практических навыков реализаций алгоритма Левенштейна и Дамерау-Левенштейна;
	
	\item проведение сравнительного анализа алгоритмов определения расстояния между строками по затратам времени и памяти;
	
	\item получение экспериментального подтверждения различий по временной эффективности алгоритмов расстояния между строками, путем измерения процессорного время с помощью разработанного программного обеспечения;
	
	\item описание и обоснование полученных результатов в отчете о выполненной лабораторной работе, выполненного как расчетно-пояснительная записка к работе. 
\end{enumerate}

