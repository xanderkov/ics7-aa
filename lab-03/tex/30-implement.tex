\chapter{Технологический раздел}

В данном разделе будут приведены требования к программному обеспечению, средства реализации и листинга кода.

\section{Требования к ПО}

Программное обеспечение должно удовлетворять следующим требованиям:
\begin{itemize}
	\item на вход подается массив целых чисел в диапозоне от $0$ до $10000$;
	\item возвращается отсортированный по возрастанию массив;
	\item в программе возможно измерение процессорного времени.
\end{itemize}

\section{Средства реализации} 
Для реализации ПО был выбран язык программирования Python\cite{python}. 

В данном языке есть все требующиеся инструменты для данной лабораторной работы.

В качестве среды разработки была выбрана среда VS Code\cite{vscode}, запуск происходил через команду python main.py.

\section{Средства замера времени}

Алгоритмы тестировались при помощи функции process\_time библиотеки time \ref{bench}. Данная команда возвращает значения процессорного времени типа int в наносекундах.

Замеры времени для каждого алгоритма проводились 100 раз.
\newpage
\begin{lstlisting}[label=bench,caption=Пример теста эффективности]
	def test_simple_mult(A, B):
	# Start the stopwatch / counter 
	t1_start = process_time() 
	for i in range(N_TEST):
	simple_mult(A, M, B, N, M)
	# Stop the stopwatch / counter
	t1_stop = process_time()
\end{lstlisting}



\section{Листинги кода}
Листинг \ref{lst:algebr} демонстрирует алгоритм сортировки подсчетом. 


\captionsetup{singlelinecheck = false, justification=raggedright}
\begin{lstlisting}[label=lst:algebr,caption=Алгоритм сортировки подсчетомя]
def counting_sort(alist, largest):
	c = [0]*(largest + 1)
	for i in range(len(alist)):
		c[alist[i]] += 1
	c[0] -= 1
	for i in range(1, largest + 1):
		c[i] += c[i - 1]
	result = [0]*len(alist)
	for x in reversed(alist):
		result[c[x]] = x
		c[x] -= 1
	return result
\end{lstlisting}


Листинг \ref{lst:winograd} --- алгоритм сортировки бинарным деревом.

\begin{lstlisting}[label=lst:winograd,caption=Алгоритм сортировки бинарным дерево]
def binary_sort(alist):
	tree = TreeNode()
	
	for i in alist:
		tree.inpurt_in_tree(i)
	
	arr = []
	tree.pre_order(arr)
	
	return arr
\end{lstlisting}	

Листинг \ref{lst:Tree} --- класс бинарного дерева.

\begin{lstlisting}[label=lst:Tree,caption=Класс бинарного дерев]
	class TreeNode:
	def __init__(self, value=None):
		self.value = value
		self.left = None
		self.right = None
	
	def inpurt_in_tree(self, elem):
		if self.value is None:
			self.value = elem
			return
		
		if elem < self.value:
			if self.left is None:
				self.left = TreeNode(elem)
			else:
				self.left.inpurt_in_tree(elem)
			
		elif elem >= self.value:
			if self.right is None:
				self.right = TreeNode(elem)
			else:
				self.right.inpurt_in_tree(elem)
	
	def pre_order(self, arr):
		if self.value:
			if self.left:
				self.left.pre_order(arr)
			if self.value: 
				arr.append(self.value)
			
			if self.right:
				self.right.pre_order(arr)
\end{lstlisting}	

\newpage
Листинг \ref{lst:winograd-optimized} --- алгоритм поразрядной сортировки.
\begin{lstlisting}[label=lst:winograd-optimized,caption=Алгоритм поразрядной сортировки]
	def radixSort(array):
    max_element = max(array)

    place = 1
    while max_element // place > 0:
        countingSort(array, place)
        place *= 10
    return array
    
\end{lstlisting}	

\section{Тестовые данные}

В таблице \ref{tbl:functional_test} приведены тесты для функций, реализующих алгоритмы сортировки. Тесты для всех сортировок пройдены \textit{успешно}.



\begin{table}[ht!]
	\begin{center}
		\captionsetup{justification=raggedleft,singlelinecheck=off}
		\caption{\label{tbl:functional_test} Функциональные тесты}
		\begin{tabular}{|c|c|c|}
			\hline
			Входной массив & Ожидаемый результат & Результат \\ 
			\hline
			$[1, 2, 3, 4, 5, 6, 7]$ & $[1, 2, 3, 4, 5, 6, 7]$  & $[1, 2, 3, 4, 5, 6, 7]$\\
			$[7, 6, 5, 4, 3, 2, 1]$  & $[1, 2, 3, 4, 5, 6, 7]$ & $[1, 2, 3, 4, 5, 6, 7]$\\
			$[9, 7, 5, 1, 4]$  & $[1, 4, 5, 7, 9]$  & $[1, 4, 5, 7, 9]$\\
			$[69]$  & $[69]$  & $[69]$\\
			$[]$  & $[]$  & $[]$\\
			\hline
		\end{tabular}
	\end{center}
\end{table}

\section{Вывод}
Было написано и протестировано программное обеспечение для решения поставленной задачи.