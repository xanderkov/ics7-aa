
\chapter*{Введение}
\addcontentsline{toc}{chapter}{Введение}

Цель лабораторной работы --- разработка, реализация, оптимизация и исследование алгоритма Копперсмита – Винограда.


Разработка и совершенствование матричных алгоритмов является важнейшей алгоритмической задачей. Непосредственное применение классического матричного умножения требует времени порядка $O(n^3)$. Однако существуют алгоритмы умножения матриц, работающие быстрее очевидного. В линейной алгебре алгоритм Копперсмита – Винограда\cite{winograd-origin}, названный в честь Д. Копперсмита и Ш. Винограда , был асимптотически самый быстрый из известных алгоритмов умножения матриц с 1990 по 2010 год. В данной работе внимание акцентируется на алгоритме Копперсмита – Винограда и его улучшениях. 

Алгоритм не используется на практике, потому что он дает преимущество только для матриц настолько больших размеров, что они не могут быть обработаны современным вычислительным оборудованием. Если матрица не велика, эти алгоритмы не приводят к большой разнице во времени вычислений. 

\newpage

Задачи данной лабораторной следующее:

\begin{enumerate}[label=\arabic*)]
	\item изучение алгоритмов перемножения матриц;
	
	\item применение оптимизации при реализации алгоритма умножения матриц Копперсмита–Винограда;
	
	\item получение практических навыков реализаций алгоритма Копперсмита–Винограда;
	
	\item проведение сравнительного анализа алгоритмов умножения матриц по затратам времени;
	
	\item получение экспериментального подтверждения различий по временной эффективности алгоритмов умножения матрица, путем измерения процессорного время с помощью разработанного программного обеспечения;
	
	\item описание и обоснование полученных результатов в отчете о выполненной лабораторной работе, выполненного как расчетно-пояснительная записка к работе. 
\end{enumerate}