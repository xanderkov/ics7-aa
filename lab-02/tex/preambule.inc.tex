\usepackage{cmap} % Улучшенный поиск русских слов в полученном pdf-файле
\usepackage[T2A]{fontenc} % Поддержка русских букв
\usepackage[utf8]{inputenc} % Кодировка utf8
\usepackage[english,russian]{babel} % Языки: русский, английский
%\usepackage{pscyr} % Нормальные шрифты
\usepackage{enumitem}


\usepackage[14pt]{extsizes}

\usepackage{graphicx}
\usepackage{multirow}

\usepackage{tikz}
\usepackage{pgfplots}
\usepackage{subcaption}
\usepackage{caption}
\captionsetup{labelsep=endash}
\captionsetup[figure]{name={Рисунок}}
\captionsetup[subtable]{labelformat=simple}
\captionsetup[subfigure]{labelformat=simple}
\renewcommand{\thesubtable}{\text{Таблица }\arabic{chapter}\text{.}\arabic{table}\text{.}\arabic{subtable}\text{ --}}
\renewcommand{\thesubfigure}{\text{Рисунок }\arabic{chapter}\text{.}\arabic{figure}\text{.}\arabic{subfigure}\text{ --}}

\usepackage{amsmath}

\usepackage{geometry}
\geometry{left=30mm}
\geometry{right=15mm}
\geometry{top=20mm}
\geometry{bottom=20mm}

\usepackage{titlesec}
\titleformat{\section}
{\normalsize\bfseries}
{\thesection}
{1em}{}
\titlespacing*{\chapter}{0pt}{-30pt}{8pt}
\titlespacing*{\section}{\parindent}{*4}{*4}
\titlespacing*{\subsection}{\parindent}{*4}{*4}

\titleformat{\chapter}{\LARGE\bfseries}{\thechapter}{20pt}{\LARGE\bfseries}
\titleformat{\section}{\Large\bfseries}{\thesection}{20pt}{\Large\bfseries}


% Настройки оглавления
\usepackage{xcolor}
\usepackage{multirow}

\usepackage[pdftex]{hyperref} % Гиперссылки
\hypersetup{hidelinks}

% Листинги 
\usepackage{listings}

\definecolor{darkgray}{gray}{0.15}

\usepackage{listings-golang}
\lstset{
	language=python, % выбор языка для подсветки
	basicstyle=\small\sffamily, % размер и начертание шрифта для подсветки кода
	numbers=left, % где поставить нумерацию строк (слева\справа)
	%numberstyle=, % размер шрифта для номеров строк
	stepnumber=1, % размер шага между двумя номерами строк
	numbersep=5pt, % как далеко отстоят номера строк от подсвечиваемого кода
	frame=single, % рисовать рамку вокруг кода
	tabsize=2, % размер табуляции по умолчанию равен 4 пробелам
	captionpos=t, % позиция заголовка вверху [t] или внизу [b]
	breaklines=true,
	breakatwhitespace=true, % переносить строки только если есть пробел
	backgroundcolor=\color{white},
	keywordstyle=\color{blue}
}


% для квадратных скобок
\usepackage{array}
\DeclareMathOperator{\rank}{rank}
\makeatletter
\newenvironment{sqcases}{%
	\matrix@check\sqcases\env@sqcases
}{%
	\endarray\right.%
}
\def\env@sqcases{%
	\let\@ifnextchar\new@ifnextchar
	\left\lbrack
	\def\arraystretch{1.2}%
	\array{@{}l@{\quad}l@{}}%
}
\makeatother



\usepackage[justification=centering]{caption} % Настройка подписей float объектов


\usepackage{csvsimple}

\newcommand{\code}[1]{\texttt{#1}}

% для матриц
\makeatletter
\renewcommand*\env@matrix[1][\arraystretch]{%
	\edef\arraystretch{#1}%
	\hskip -\arraycolsep
	\let\@ifnextchar\new@ifnextchar
	\array{*\c@MaxMatrixCols c}}
\makeatother

\usepackage{amsfonts}
\usepackage{graphicx}
\newcommand{\img}[3] {
	\begin{figure}[h!]
		\center{\includegraphics[height=#1]{inc/img/#2}}
		\caption{#3}
		\label{img:#2}
	\end{figure}
}
\newcommand{\boximg}[3] {
	\begin{figure}[h]
		\center{\fbox{\includegraphics[height=#1]{inc/img/#2}}}
		\caption{#3}
		\label{img:#2}
	\end{figure}
}

