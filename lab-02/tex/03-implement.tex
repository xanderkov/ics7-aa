\chapter{Технологический раздел}

В данном разделе будут приведены требования к программному обеспечению, средства реализации и листинга кода.

\section{Требования к ПО}

Программное обеспечение должно удовлетворять следующим требованиям:
\begin{itemize}
	\item программа получает на вход с клавиатуры две матрицы размеров в пределах $10000 \times 10000$ либо получает два числа --- размерности матрицы в пределах $10000$;
	\item программа выдает матрицу --- произведение двух полученных матриц;
	\item в программе возможно измерение процессорного времени.
\end{itemize}

\section{Средства реализации} 
Для реализации ПО был выбран язык программирования Python \cite{python}. 

В данном языке есть все требующиеся инструменты для данной лабораторной работы.

В качестве среды разработки была выбрана среда VS Code \cite{vscode}, запуск происходил через команду python main.py.

\section{Средства замера времени}

Замеры времени выполнения реализаций алгоритма будут проводиться при помощи функции process\_time \cite{test} библиотеки time. Данная команда возвращает значения процессорного времени типа int в наносекундах.

Замеры времени для каждой реализации алгоритма и для каждого комплекта входных данных проводились 100 раз.

\begin{lstlisting}[label=bench,caption=Пример замера затраченного времени]
	def test_simple_mult(A, B):
	# Start the stopwatch / counter 
	t1_start = process_time() 
	for i in range(N_TEST):
	simple_mult(A, M, B, N, M)
	# Stop the stopwatch / counter
	t1_stop = process_time()
\end{lstlisting}



\section{Реализация алгоритмов}
Листинг \ref{lst:algebr} демонстрирует реализацию классического алгоритма умножения. 


\captionsetup{singlelinecheck = false, justification=raggedright}
\begin{lstlisting}[label=lst:algebr,caption=Классический алгоритм умножения]
def simple_mult(mat1, m, mat2, n, q):
	res = np.zeros([m, q])
	for i in range(m):
		for j in range(q):
			for k in range(n):
				res[i][j] = res[i][j] + mat1[i][k] * mat2[k][j]
	return res
\end{lstlisting}

\newpage
Листинг \ref{lst:winograd} демонстрирует умножение матриц реализации алгоритмом Винограда.

\begin{lstlisting}[label=lst:winograd,caption=Алгоритм умнложения Виноградом]
def precompile\_rows\_win(mat, n, m):
	mh = np.zeros([n])
	for i in range(n):
		for j in range(m // 2):
			mh[i] = mh[i] + mat[i][j * 2] * mat[i][j * 2 + 1]
	return mh

def precompile\_cols\_win(mat, n, m):
	mv = np.zeros([m])
	
	for i in range(m):
		for j in range(n // 2):
			mv[i] = mv[i] + mat[j * 2][i] * mat[j * 2 + 1][i]
	return mv

def winograd\_mult(A, m, B, n, q):
	res = np.zeros([m, q])
	mh = precompile\_rows\_win(A, m, n)
	mv = precompile\_cols\_win(B, n, q)
	for i in range(m):
		for j in range(q):
		res[i][j] = -mh[i] - mv[j]
		for k in range(n // 2):
			res[i][j] = res[i][j] + (A[i][k*2] + B[k*2+1][j])*(A[i][k*2+1] + B[k*2][j])
	if n \% 2 != 0:
		for i in range(n):
			for j in range(m):
				res[i][j] = res[i][j] + A[i][n-1]*B[n-1][j]
	return res

\end{lstlisting}	

\newpage
Листинг \ref{lst:winograd-optimized} демонстрирует умножение реализации оптимизированным алгоритмом Винограда.
\begin{lstlisting}[label=lst:winograd-optimized,caption=Оптимизированный алгоритм умножения Виноградом]
def precompile_rows_win_opt(mat, n, m):
	mh = np.zeros([n])
	opt = m // 2
	for i in range(n):
		for j in range(opt):
			t = j << 1
			mh[i] += mat[i][t] * mat[i][t + 1]
	return mh

def precompile_cols_win(mat, n, m):
	mv = np.zeros([m])
	
	opt = n // 2
	for i in range(m):
		for j in range(opt):
			t = j << 1
			mv[i] += mat[t][i] * mat[t + 1][i]
	return mv

def winograd_mult_opt(A, m, B, n, q):
	res = np.zeros([m, q])
	mh = precompile_rows_win(A, m, n)
	mv = precompile_cols_win(B, n, q)
	
	opt = n // 2
	for i in range(m):
		for j in range(q):
			res[i][j] = -mh[i] - mv[j]
			for k in range(n // 2):
				t = k << 1
				res[i][j] += (A[i][t] + B[t+1][j])*(A[i][t+1] + B[t][j])
	if n % 2 != 0:
		for i in range(n):
			for j in range(m):
				res[i][j] += A[i][n-1]*B[n-1][j]
	return res
\end{lstlisting}


\captionsetup{singlelinecheck = false, justification=raggedleft}



\section{Тестовые данные}

В таблице \ref{tbl:functional_test} представлены тестовые данные. Применена методология черного ящика. Все тесты пройдены успешно.

\begin{table}[ht!]
	\begin{center}
			\captionsetup{justification=raggedright,singlelinecheck=off}
			\caption{\label{tbl:functional_test} Функциональные тесты}
			\begin{tabular}{|c|c|c|}
				\hline
				1-ая матрица & 2-ая матрица & Ожидаемый результат 
				\\ \hline
				$\begin{pmatrix}
					\\
				\end{pmatrix}$ &
				$\begin{pmatrix}
					\\
				\end{pmatrix}$ &
				Сообщение об ошибке
				
				\\ \hline
				$\begin{pmatrix}
					\\
				\end{pmatrix}$ &
				$\begin{pmatrix}
					1 & 1\\
					1 & 1
				\end{pmatrix}$ &
				Сообщение об ошибке
				
				\\ \hline
				$\begin{pmatrix}
					0 & 1
				\end{pmatrix}$ &
				$\begin{pmatrix}
					0 & 1
				\end{pmatrix}$ &
				Сообщение об ошибке 
				
				\\ \hline
				$\begin{pmatrix}
					5
				\end{pmatrix}$ &
				$\begin{pmatrix}
					5
				\end{pmatrix}$ &
				$\begin{pmatrix}
					25
				\end{pmatrix}$
				
				\\ \hline
				$\begin{pmatrix}
					1 & 2 & 3\\
					4 & 5 & 6\\
					7 & 8 & 9
				\end{pmatrix}$ &
				$\begin{pmatrix}
					1 & 2 & 3\\
					4 & 5 & 6\\
					7 & 8 & 9
				\end{pmatrix}$ &
				$\begin{pmatrix}
					30 & 36 & 42\\
					66 & 81 & 96\\
					102 & 126 & 150
				\end{pmatrix}$ 
				
				\\ \hline
				$\begin{pmatrix}
					4 & 3 \\
					2 & 1 
				\end{pmatrix}$ &
				$\begin{pmatrix}
					1 & 2\\
					3 & 4
				\end{pmatrix}$ &
				$\begin{pmatrix}
					13 & 20 \\
					5 & 8 
				\end{pmatrix}$                  
				
				\\ \hline
				$\begin{pmatrix}
					1 & 2 & 3
				\end{pmatrix}$ &
				$\begin{pmatrix}
					1\\
					2\\
					3
				\end{pmatrix}$ &
				$\begin{pmatrix}
					14
				\end{pmatrix}$ 
				
				\\ \hline
				$\begin{pmatrix}
					1 & 2 & 3 \\
					4 & 5 & 6 \\
					7 & 8 & 9
				\end{pmatrix}$ &
				$\begin{pmatrix}
					1 & 4\\
					2 & 5\\
					3 & 6
				\end{pmatrix}$ &
				$\begin{pmatrix}
					14 & 32 \\
					32 & 77 \\
					50 & 122
				\end{pmatrix}$                  
				\\ \hline
			\end{tabular}
	\end{center}
\end{table}

\section*{Вывод}
Было написано и протестировано программное обеспечение для решения поставленной задачи.