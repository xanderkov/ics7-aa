\chapter{Аналитический раздел}\label{analyth}

В этом разделе будут представлены описания алгоритмов умножения матриц и алгоритм Винограда.


\section{Применение математического подхода}
Даны матрицы, $A \in \mathbb{R}^{m \times n}$, $B \in \mathbb{R}^{n \times p}$, произведение матриц, $C = A \times B$, каждый элемент которой вычисляется согласно формуле
\begin{equation}\label{math:mtx-arythm} 
	c_{i,j} = \sum_{n}^{k = 1}a_{i, k}\cdot b_{k, j}, \text{  где } i = \overline{1, m}, j = \overline{1, p} 
\end{equation}

\noindent 

Стандартный алгоритм умножения матриц реализует формулу (\ref{math:mtx-arythm}). 

Операция умножения двух матриц выполнима только в том случае, если число столбцов в первом сомножителе равно числу строк во втором.  

\section{Алгоритм Винограда}

Алгоритм Винограда \cite{YV} --- алгоритм умножения матриц.
Рассмотрим два вектора $U = (u_1, u_2, u_3, u_4)$, $W = (w_1, w_2, w_3, w_4)$. Их скалярное произведение равно $U \cdot W = u1w1 + u2w2 + u3w3 + u4w4$, что эквивалентно:
	$$
	U\cdot W = (u_1 + w_2)(w_1 + u_2) + (u_4 + w_3)(w_4 + u_3) - u_1u_2 - u_3u_4 - w_1w_2 - w_3w_4 $$
	
За счет предварительной обработки последних 4 слагаемых можно получить прирост производительности.

Стоит упомянуть, что при нечетном количестве столбцов матриц нужно дополнительно добавить произведение последних элементов соответствующих строки и столбца к скалярному произведению строки и столбца.

\section*{Вывод}
Была выявлена основная особенность подхода Винограда --- идея предварительной обработки данных. Разница во времени выполнения реализаций этих двух алгоритмов будет экспериментально вычислена в исследовательском разделе. 