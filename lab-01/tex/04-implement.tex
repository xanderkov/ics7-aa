\chapter{Технологический раздел}

\section{Требования к ПО}
Программа должна отвечать следующим требованиям:
\begin{itemize}
	\item ПО корректно реагирует на любые действия пользователя;
	\item ПО возвращает полученное расстояние;
	\item ПО принимает текстовые данные в любой раскладке.
	\item Время отклика программы на любое действие пользователя должно быть приемлемым.
	\item ПО корректно высчитывает время своей работы 
\end{itemize}
\section{Средства реализации}
Для реализации ПО был выбран язык программирования Golang\cite{golang}. Данный язык предоставляет следующие возможности:
\begin{itemize}
	\item Язык компилируемый, то есть на выходе будет получен исполняемый файл, что позволит точно измерить время работы программы.
	\item Средства объектно-ориентированного программирования ограничиваются интерфейсами. Это позволяет создавать абстракции, при этом оставаясь высокоскоростным языком.
	\item Статическая типизация. Позволяет избежать ошибок, допущенных по невнимательности, упрощает чтение и понимание кода.
	\item Обширная стандартная библиотека. Позволяет использовать уже написанные шаблоны, что уменьшает количество кода, которое необходимо написать. Поддерживает библиотеки C и C++  
\end{itemize}

В качестве среды разработки была выбрана среда VS Code\cite{rune}, запуск происходил через команду go run main.go.

\section{Листинги кода}
\subsection{Реализация алгоритмов}

В качестве представления строковых данных был выбран тип rune\cite{rune} -- псевдоним для типа int32. 

В листингах \ref{leven-recursive} - \ref{damer-lev-rec} приведены реализации алгоритмов, описанных в разделе \ref{analyth}.


\begin{lstlisting}[label=leven-recursive,caption=Программный код нахождения расстояния Дамерау\,--\,Левенштейна итеративно]
func CountDamNoRec(src, dest string) (int, MInt) {
	var (n, m, dist, shortDist, transDist int)
	
	srcRune, destRune := []rune(src), []rune(dest)
	
	n, m = len(srcRune), len(destRune)
	
	distMat := makeMatrix(n, m)
	
	for i := 1; i < n + 1; i++ {
		for j := 1; j < m + 1; j++ {
			insDist := distMat[i][j - 1] + 1
			delDist := distMat[i - 1][j] + 1
			
			match := 1
			if src[i - 1] == dest[j - 1] {
				match = 0
			}
			eqDist := distMat[i - 1][j - 1] + match
			
			transDist = -1
			if i > 1 && j > 1 {
				transDist = distMat[i - 2][j - 2] + 1
			}
			
			if transDist != -1 && srcRune[i - 1] == destRune[j - 2] && 
			srcRune[i - 2] == destRune[j - 1] {
				dist = getMinOfValues(insDist, delDist, eqDist, transDist)
			} else {
				dist = getMinOfValues(insDist, delDist, eqDist)
			}
			distMat[i][j] = dist
		}
	}
	
	shortDist = distMat[n][m]
	
	return shortDist, distMat
}
\end{lstlisting}

\begin{lstlisting}[label=damer-lev-iter,caption=Программный код нахождения расстояния Дамерау\,--\,Левенштейна рекурсивно]
func _countDamRecElem(src, dest []rune, i, j int) int {
	if (getMinOfValues(i, j) == 0) {
		return getMaxOf2Values(i, j)
	}
	
	match := 1
	if (src[i - 1] == dest[j - 1]) {
		match = 0
	}
	
	insert := _countDamRecElem(src, dest, i, j - 1) + 1
	delete := _countDamRecElem(src, dest, i - 1, j) + 1
	replace := match+_countDamRecElem(src, dest, i - 1, j - 1)
	
	transpose := -1
	
	if i > 1 && j > 1 {
		transpose = _countDamRecElem(src, dest, i - 2, j - 2) + 1
	}
	
	min := 0
	if transpose != -1 && src[i - 1] == dest[j - 2] 
	&& src[i - 2] == dest[j - 1] {
		min = getMinOfValues(insert, delete, replace, transpose)
	} else {
		min = getMinOfValues(insert, delete, replace)
	}
	return min
}

func CountDamRecNoCache(src, dest string) int {
	
	srcRune, destRune := []rune(src), []rune(dest)
	n, m := len(srcRune), len(destRune)
	
	return _countDamRecElem(srcRune, destRune, n, m)
}
\end{lstlisting}

\begin{lstlisting}[label=damer-lev-rec,caption=Программный код нахождения расстояния Дамерау\,--\,Левенштейна рекурсивно с кэшем]	
func _countDamRecElemCache(src, dest []rune, i, j int, distMat MInt) int {
	if (getMinOfValues(i, j) == 0) {
		return getMaxOf2Values(i, j)
	}
	
	if distMat[i][j] != -1 {
		return distMat[i][j]
	}
	
	match := 1
	if (src[i - 1] == dest[j - 1]) {
		match = 0
	}
	
	insert := _countDamRecElemCache(src, dest, i, j - 1, distMat) + 1
	delete := _countDamRecElemCache(src, dest, i - 1, j, distMat) + 1
	replace := match + _countDamRecElemCache(src, dest, i - 1, j - 1, distMat)
	
	transpose := -1
	
	if i > 1 && j > 1 {
		transpose = _countDamRecElemCache(src, dest, i - 2, j - 2, distMat) + 1
	}
	
	min := 0
	if transpose != -1 && src[i - 1] == dest[j - 2] 
	&& src[i - 2] == dest[j - 1] {
		min = getMinOfValues(insert, delete, replace, transpose)
	} else {
		min = getMinOfValues(insert, delete, replace)
	}
	
	distMat[i][j] = min
	return distMat[i][j]
	
}

func CountDamRecCache(src, dest string) int {
	srcRune, destRune := []rune(src), []rune(dest)
	n, m := len(srcRune), len(destRune)
	
	distMat := makeMatrixInf(n, m)
	
	return _countDamRecElemCache(srcRune, destRune, n, m, distMat)
}
\end{lstlisting}

\subsection{Утилиты}
В листингах \ref{min3} - \ref{i-matrix} приведены используемые утилиты.
\begin{lstlisting}[label=min3,caption=Функция нахождения минимума из N целых чисел]
	func getMinOfValues(values ...int) int {
		min := values[0]
		
		for _, i := range values {
			if min > i {
				min = i
			}
		}
		
		return min
	}
	
\end{lstlisting}

\begin{lstlisting}[label=max2,caption=Функция нахождения максимума из двух целых чисел]
	func getMaxOf2Values(v1, v2 int) int {
		if v1 < v2 {
			return v2
		}
		return v1
	}
\end{lstlisting}


\begin{lstlisting}[label=i-matrix,caption=Определение типа целочисленной матрицы; его инициализация и вывод]
	type MInt [][]int
	
	func makeMatrix(n, m int) MInt {
		matrix := make(MInt, n + 1)
		
		for i := range matrix {
			matrix[i] = make([]int, m + 1)
		}
		
		for i := 0; i < m + 1; i++ {
			matrix[0][i] = i
		}
		
		for i := 0; i < n + 1; i++ {
			matrix[i][0] = i
		}
		return matrix
	}
	

	func (mat MInt) PrintMatrix() {
		for i := 0; i < len(mat); i++ {
			for j := 0; j < len(mat[0]); j++ {
				fmt.Printf("%3d ", mat[i][j])
			}
			fmt.Printf("\n")
		}
}

\end{lstlisting}

\section{Тестовые данные}

\begin{table}[h!]
	\begin{center}
		
		\begin{tabular}{||c c c | c c c||} 
			\hline
			№ & $S_1$ & $S_2$ & DLIter & DLRec & DLRecCache \\
			[0.5ex] 
			\hline\hline
			1 & << >> & << >> & 0 & 0 & 0 \\
			2 & <<book>> & <<bosk>> & 1 & 1 & 1 \\ 
			3 & <<book>> & <<back>> & 2 & 2 & 2 \\ 
			4 & <<book>> & <<bacc>> & 3 & 3 & 3 \\ 
			5 & <<aboba>> & <<acacb>> & 4 & 4 & 4 \\ 
			6 & <<дверь>> & <<деврь>> & 1 & 1 & 1 \\   
			6 & <<дверь>> & <<дверь>> & 1 & 1 & 1 \\   
			\hline
		\end{tabular}
	\end{center}
\end{table}



\section{Вывод}
На основе схем из конструкторского раздела были разработаны программные реализации требуемых алгоритмов.