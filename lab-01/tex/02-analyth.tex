\chapter{Аналитический раздел}\label{analyth}

\section{Расстояние Левенштейна}\label{defs}

Редакторское расстояние (расстояние Левенштейна) --- это минимальное количество операций вставки, удаления и замены, необходимых для превращения одной строки в другую. Каждая редакторская операция имеет цену (штраф). 
В общем случае, имея на входе строку, $X = x_1x_2 \dots x_n$ , и, $Y = y_1y_2 \dots y_n$, расстояние между ними можно вычислить с помощью операций:
\begin{itemize}
	\item ${\text{delete}(u, \varepsilon) = \delta}$
	\item $\text{insert}(\varepsilon, v) = \delta$
	\item $\text{replace}(u, v) = \alpha(u, v) \leq 0$  $($здесь, $\alpha(u, u) = 0$ для всех $u).$
\end{itemize}

Необходимо найти последовательность замен с минимальным суммарным штрафом. Далее, цена вставки и удаления будет считаться равной 1.
 Пусть даны строки: s1 = s1[1..L1], s2 = s2[1..L2], s1[1..i] --- подстрока s1 длинной i, начиная с 1-го символа, s2[1..j] - подстрока s2 длинной j, начиная с 1-го символа. Расстояние Левентштейна посчитывается следующей формулой:
 
\begin{equation}\label{form:lev}
	D(s1[1..i], s2[1..j]) =
	\begin{cases}
		0,       & i = 0, j =0,\\
		i  & i > 0, j =0, \\
		j, & j > 0, i = 0, \\
		\min(D(s1[1..i], s2[1..j-1]) + 1\\, 
		\min(D(s1[1..i-1], s2[1..j]) + 1, \\,
		\min(D(s1[1..i-1], s2[1..j]) + 1 \\ + 
		\begin{sqcases}
			0, & если s1[i]=s2[j] \\
			1 & иначе
		\end{sqcases}
	\end{cases}
\end{equation}


\section{Расстояние Дамерау\,--\,Левенштейна}
Расстояние Дамерау\,--\,Левенштейна --- модификация расстояния Левенштейна, добавляющая транспозицию к редакторским операциям, предложенными Левенштейном (см. \ref{defs}). изначально алгоритм разрабатывался для сравнения текстов, набранных человеком (Дамерау показал, что 80\% человеческих ошибок при наборе текстов составляют перестановки соседних символов, пропуск символа, добавление нового символа, и ошибка в символе. Поэтому метрика Дамерау\,--\,Левенштейна часто используется в редакторских программах для проверки правописания). 

Используя условные обозначения, описанные в разделе \ref{defs}, рекурсивная формула для нахождения расстояния Дамерау\,--\,Левенштейна, $f(i, j)$ ,между подстроками, $x_1 \dots x_i$ и, $y_1 \dots y_j$, имеет следующий вид:
\begin{equation}\label{form:d-lev}
	f_{X,Y}(i, j) = 
	\begin{cases}
		\delta_i & j = 0, \\
		\delta_j & i = 0, \\ 
		\min 
		\begin{cases}
			\alpha(x_i, y_i) + f_{X,Y}(i - 1, j - 1) \\
			\delta + f_{X,Y}(i - 1, j) \\
			\delta + f_{X,Y}(i, j - 1) \\
			\begin{sqcases}
				\delta + f_{X,Y}(i - 2, j - 2) & i, j > 1 и x_i = y_{j - 1} и x_{i - 1} = y_j \\
				\infty & \text{иначе;}
			\end{sqcases}
		\end{cases} \\
	\end{cases}
\end{equation}

\section{Рекурсивная формула}
Используя условные обозначения, описанные в разделе \ref{form:d-lev}, рекурсивная формула для нахождения расстояния Дамерау\,--\,Левенштейна $f(i, j)$ между подстроками, $x_1 \dots x_i$, и, $y_1 \dots y_j$, имеет следующий вид:
\begin{equation}
	f_{X,Y}(i, j) = 
	\begin{cases}
		\delta_i & j = 0, \\
		\delta_j & i = 0, \\ 
		\min 
		\begin{cases}
			\alpha(x_i, y_i) + f_{X,Y}(i - 1, j - 1) \\
			\delta + f_{X,Y}(i - 1, j) \\
			\delta + f_{X,Y}(i, j - 1) \\
			\begin{sqcases}
				\delta + f_{X,Y}(i - 2, j - 2) & i, j > 1 и x_i = y_{j - 1} и x_{i - 1} = y_j \\
				\infty & \text{иначе;}
			\end{sqcases}
		\end{cases} \\
	\end{cases}
\end{equation}

$f_{X,Y}$ --- редакционное расстояние между двумя подстроками --- первыми $i$ символами строки $X$ и первыми $j$ символами строки $Y$. Можно вывести следующие утверждения:
\begin{itemize}
	\item Если редакционное расстояние нулевое, то строки равны:\newline
	$f_{X, Y} = 0 \Rightarrow X = Y$ 
	\item Редакционное расстояние симметрично:\newline
	$f_{X, Y} = f_{Y, X}$ 
	\item Максимальное значение $f_{X, Y}$ --- размерность более длинной строки:\newline
	$f_{X, Y} \leq max(|X|, |Y|)$ 
	\item Минимальное значение $f_{X, Y}$ --- разность длин обрабатываемых строк:\newline
	$f_{X, Y} \geq abs(|X| - |Y|)$ 
	\item Аналогично свойству треугольника, редакционное расстояние между двумя строками не может быть больше чем редакционные расстояния каждой из этих строк с третьей:\newline
	$f_{X, Y} \leq f_{X, Z} + f_{Z, Y}$ 
\end{itemize} 


\section{Матрица расстояний}
В алгоритме нахождения редакторского расстояния Дамерау\,--\,Левенштейна возможно использование матрицы расстояний.

Пусть, $C_{0 \dots |X|, 0 \dots |Y|}$, --- матрица расстояний, где, $C_{i, j}$ --- минимальное количество редакторских операций, необходимое для преобразования подстроки, $x_1\dots x_i$, в подстроку, $y_1 \dots y_j$. Матрица заполняется следующим образом:
\begin{equation}\label{matrix}
	C{i, j} = 
	\begin{cases}
		i & j = 0, \\
		j & i = 0, \\
		\min 
		\begin{cases}
			C_{i - 1, j - 1} + \alpha(x_i, y_i), \\
			C_{i - 1, j} + 1, \\
			C_{i, j - 1} + 1)
		\end{cases}
	\end{cases}
\end{equation}

При решении данной задачи используется ключевая идея динамического программирования --- чтобы решить поставленную задачу, требуется разбить на отдельные части задачи (подзадачи), после чего объединить решения подзадач в одно общее решение. Здесь небольшие подзадачи --- это заполнение ячеек таблицы с индексами, $i < |X|, j < |Y|$. После заполнения всех ячеек матрицы в ячейке, $C_{|X|, |Y|}$, будет записано искомое расстояние.



\section{Рекурсивный алгоритм расстояния Дамерау\,--\,Левенштейна с мемоизацией}
При реализации рекурсивного алгоритма используется мемоизация --- сохранение результатов выполнения функций для предотвращения повторных вычислений. Отличие от формулы \ref{matrix} состоит лишь в начальной инициализации матрицы флагом $\infty$, который сигнализирует о том, была ли обработана ячейка. В случае если ячейка была обработана, алгоритм переходит к следующему шагу. 


\section{Вывод}
Были рассмотрены обе вариации алгоритма редакторского расстояния (Левенштейна и Дамерау\,--\,Левенштейна). Также были приведены разные способы реализации этих алгоритмов такие как: рекурсивный, итеративный и рекурсивный с мемоизацией. Итеративный может быть реализован с помощью парадигм динамического программирования или матрицей расстояния. Рекурсивный с мемоизацией позволяет ускорить обычный рекурсивный алгоритм за счет матрицы, в которой промежуточные подсчеты.