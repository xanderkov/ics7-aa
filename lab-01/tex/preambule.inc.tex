\usepackage{cmap}
\usepackage[T2A]{fontenc} 
\usepackage[utf8]{inputenc}
\usepackage[english,russian]{babel}


\usepackage{enumitem}
\usepackage[14pt]{extsizes}

\usepackage{graphicx}
\usepackage{multirow}

\usepackage{tikz}
\usepackage{pgfplots}

\usepackage{caption}
\captionsetup{labelsep=endash}
\captionsetup[figure]{name={Рисунок}}

\usepackage{amsmath}

\usepackage{geometry}
\geometry{left=30mm}
\geometry{right=15mm}
\geometry{top=20mm}
\geometry{bottom=20mm}

\usepackage{titlesec}
\titleformat{\section}
{\normalsize\bfseries}
{\thesection}
{1em}{}
\titlespacing*{\chapter}{0pt}{-30pt}{8pt}
\titlespacing*{\section}{\parindent}{*4}{*4}
\titlespacing*{\subsection}{\parindent}{*4}{*4}

\usepackage{setspace}
\onehalfspacing % Полуторный интервал

\frenchspacing
\usepackage{indentfirst} % Красная строка

\usepackage{titlesec}
\usepackage{xcolor}
% Названия глав
\titleformat{\section}{\normalsize\textmd}{\thesection}{1em}{}

\definecolor{gray75}{gray}{0.75}

\newcommand{\hsp}{\hspace{20pt}} % длина линии в 20pt

\titleformat{\chapter}[hang]{\Huge}{\thechapter\hsp\textcolor{gray75}{|}\hsp}{0pt}{\Huge\textmd}

\titleformat{\section}{\Large}{\thesection}{20pt}{\Large\textmd}
\titleformat{\subsection}{\Large}{\thesubsection}{20pt}{\Large\textmd}
\titleformat{\subsubsection}{\normalfont\textmd}{}{0pt}{}

% Настройки оглавления

\addtocontents{toc}{\setcounter{tocdepth}{2}}
\addtocontents{toc}{\setcounter{secnumdepth}{1}}

\usepackage{tocloft,lipsum,pgffor}

\addtocontents{toc}{~\hfill\textnormal{Страница}\par}

\usepackage{minitoc}

\renewcommand{\cfttoctitlefont}{\Huge\textmd}

\renewcommand{\cftpartfont}{\normalfont\textmd}

\renewcommand{\cftchapfont}{\normalfont\normalsize}
\renewcommand{\cftsecfont}{\normalfont\normalsize}
\renewcommand{\cftsubsecfont}{\normalfont\normalsize}
\renewcommand{\cftsubsubsecfont}{\normalfont\normalsize}

\renewcommand{\cftchapleader}{\cftdotfill{\cftdotsep}}

\usepackage{listings}
\usepackage{xcolor}

\bibliographystyle{gost-numeric.bbx}
\usepackage[backend=biber,
sorting=none,
]{biblatex} 


\addbibresource{ref-lib.bib} % База библиографии

\usepackage[pdftex]{hyperref} % Гиперссылки
\hypersetup{hidelinks}

% Листинги 
\usepackage{listings}

\definecolor{darkgray}{gray}{0.15}

\usepackage{listings-golang}
\lstset{ % add your own preferences
	frame=single,
	basicstyle=\footnotesize\ttfamily,
	identifierstyle=\color{darkgray},
	keywordstyle=\color{black},
	numbers=left,
	numbersep=5pt,
	numberstyle=\tiny,
	showstringspaces=false, 
	captionpos=t,
	tabsize=4,
	language=Golang
}

% какой то сложный кусок со стак эксчейндж для квадратных скобок
\usepackage{array}
\DeclareMathOperator{\rank}{rank}
\makeatletter
\newenvironment{sqcases}{%
	\matrix@check\sqcases\env@sqcases
}{%
	\endarray\right.%
}
\def\env@sqcases{%
	\let\@ifnextchar\new@ifnextchar
	\left\lbrack
	\def\arraystretch{1.2}%
	\array{@{}l@{\quad}l@{}}%
}
\makeatother

\usepackage{amsmath}

\usepackage{geometry}
\geometry{left=30mm}
\geometry{right=15mm}
\geometry{top=20mm}
\geometry{bottom=20mm}

\usepackage{titlesec}
\titleformat{\section}
{\normalsize\bfseries}
{\thesection}
{1em}{}
\titlespacing*{\chapter}{0pt}{-30pt}{8pt}
\titlespacing*{\section}{\parindent}{*4}{*4}
\titlespacing*{\subsection}{\parindent}{*4}{*4}

\usepackage{setspace}
\onehalfspacing % Полуторный интервал

\frenchspacing
\usepackage{indentfirst} % Красная строка

\usepackage{titlesec}
\titleformat{\chapter}{\LARGE\bfseries}{\thechapter}{20pt}{\LARGE\bfseries}
\titleformat{\section}{\Large\bfseries}{\thesection}{20pt}{\Large\bfseries}

\usepackage[justification=centering]{caption} % Настройка подписей float объектов


\usepackage{csvsimple}

\newcommand{\code}[1]{\texttt{#1}}

