\chapter*{Введение}

Нахождение редакционного расстояния ~ одна из задач компьютерной лингвистики, которая находит применение в огромном количестве областей, начиная от предиктивных систем набора текста и заканчивая разработкой искусственного интеллекта. Впервые задачу поставил советский ученый В. И. Левенштейн \cite{Lev1965}, впоследствии её связали с его именем. В данной работе будут рассмотрены алгоритмы редакционного расстояния Левенштейна и расстояние Дамерау \,--\, Левенштейна \cite{damerau}.

Расстояния Левенштейна --- метрика, измеряющая разность двух строк символов, определяемая в количестве редакторских операций(а именно удаления, вставки и замены), требуемых для преобразования одной последовательности в другую.  Расстояние Дамерау -- Левенштейна ~ модификация, добавляющая к редакторским операциям транспозицию, или обмен двух соседних символов местами.
Алгоритмы имеют некоторое количество модификаций, позволяющих эффективнее решать поставленную задачу. В данной работе будут предложены реализации алгоритмов, использующие парадигмы динамического программирования.

Цель лабораторной работы -- получить навыки динамического программирования. 
Задачами лабораторной работы являются изучение и реализация алгоритмов Левенштейна и Дамерау \,--\,Левенштейна, применение парадигм динамического программирования при реализации алгоритмов и сравнительный анализ алгоритмов на основе экспериментальных данных.

В данной лабораторной работе будут рассмотрены разные реализации данных алгоритмов нахождения редакторских расстояний. Такие как: итеративный, рекурсивный и рекурсивный с кэшем. 

Также будут приведены сравнения реализации по времени и памяти.