\chapter{Конструкторский раздел}
В данном разделе представлены схемы реализуемых алгоритмов и их модификации.
\section{Матричные итерационные алгоритмы}
На рисунке \ref{fig:damerau-Page-1} изображена схема алгоритма нахождения расстояния Дамерау \,--\,Левенштейна итеративно  с использованием матрицы расстояний.
\section{Модификация матричных алгоритмов}
Мемоизация - это прием сохранения промежуточных результатов, которые могут еще раз понадобиться в ближайшее время, чтобы избежать их повторного вычисления. 
Матричный алгоритм нахождения расстояния Дамерау\,--\,Левенштейна может быть модифицирован, используя мемоизацию --- достаточно инициализировать матрицу значением $\infty$, которое будет рассмотрено в качестве флага.
На рисунке \ref{fig:damerau-mem} изображена схема алгоритма, использующая этот прием.
\section{Рекурсивные алгоритмы}
На рисунке \ref{fig:damerau-rec} изображена схема рекурсивного алгоритма  нахождения расстояния Дамерау\,--\,Левенштейна.
\newpage 


\begin{figure}[!ht]
	\begin{center}
		\includegraphics[scale=0.5]{./assets/d\_l\_iter.pdf}
	\end{center}
	
	\caption{Схема итерационного алгоритма расстояния Дамерау\,--\,Левенштейна с заполнением матрицы расстояний}
	\label{fig:damerau-Page-1}
\end{figure}

\begin{figure}[!ht]
	\begin{center}
		\includegraphics[scale=0.5]{./assets/d\_l\_rec.pdf}
	\end{center}
	
	\caption{Схема рекурсивного алгоритма расстояния Дамерау\,--\,Левенштейна с мемоизацией}
	\label{fig:damerau-mem}
\end{figure}

\begin{figure}[!ht]
	\begin{center}
		\includegraphics[scale=0.5]{assets/d\_l\_recm.pdf}
	\end{center}
	
	\caption{Схема рекурсивного алгоритма расстояния Дамерау\,--\,Левенштейна}
	\label{fig:damerau-rec}
\end{figure}



\section{Вывод}
На основе формул и теоретических данных, полученных в аналитическом разделе, были спроектированы схемы алгоритмов.