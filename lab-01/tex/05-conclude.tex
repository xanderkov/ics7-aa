\chapter*{Заключение}
\addcontentsline{toc}{chapter}{Заключение}

В рамках лабораторной работы были:

\begin{itemize}
	\item рассмотрены три алгоритма нахождения редакторского расстояния Дамерау\,--\,Левенштейна и одно Левенштейна;
	\item в аналитическом разделе были изучены смысловые различия между алгоритмами и их формульное представление;
	\item в рамках конструкторского раздела были получены схемы алгоритмов;
	\item в технологическом разделе был выбран язык программирования и представлена реализация на нем, также были приведены тестовые данные;
	\item в исследовательской части были сравнены алгоритмы по скорости и по памяти. Самым эффективным по времени оказался итеративный алгоритм. Самым эффективным по памяти --- рекурсивный алгоритм. 
\end{itemize}

По итогу реализации алгоритма поиска редакторского расстояния Дамерау\,--\,Левенштейна итеративным способом оказался быстрее остальных алгоритмов на 38 \% при длине слова в 240 символов, то есть на 0.2 секунды.
 
В ходе лабораторной работы получены навыки динамического программирования, реализованы и изученные алгоритмы нахождения редакторского расстояния. Цель работы достигнута.