\chapter*{Заключение}

В рамках лабораторной работы были рассмотрены три алгоритма нахождения  расстояния Дамерау \,--\,Левенштейна. Во время аналитического изучения алгоритмов были выявлены смысловые различия между алгоритмами.
Самая оптимальная реализация по памяти -- рекурсивный алгоритм, самая оптимальная реализация по времени -- итеративный алгоритм, использующий таблицу расстояний. Для языков, где возможна передача указателя на массивы, самым эффективным по времени и по памяти будет алгоритм, использующий мемоизацию. 
В ходе лабораторной работы получены навыки динамического программирования, реализованы изученные алгоритмы.